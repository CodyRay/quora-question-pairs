% cs434-sp17 final paper, hoeftc and gillenp
% based on nips_2017.tex

\documentclass{article}

% if you need to pass options to natbib, use, e.g.:
% \PassOptionsToPackage{numbers, compress}{natbib}
% before loading nips_2017
%
% to avoid loading the natbib package, add option nonatbib:
% \usepackage[nonatbib]{nips_2017}

% print author's names
\usepackage[final]{nips_2017}

\usepackage[utf8]{inputenc} % allow utf-8 input
\usepackage[T1]{fontenc}    % use 8-bit T1 fonts
\usepackage{hyperref}       % hyperlinks
\usepackage{url}            % simple URL typesetting
\usepackage{booktabs}       % professional-quality tables
\usepackage{amsfonts}       % blackboard math symbols
\usepackage{nicefrac}       % compact symbols for 1/2, etc.
\usepackage{microtype}      % microtypography

\title{Finding Quora questions pairs with neural networks}

% The \author macro works with any number of authors. There are two
% commands used to separate the names and addresses of multiple
% authors: \And and \AND.
%
% Using \And between authors leaves it to LaTeX to determine where to
% break the lines. Using \AND forces a line break at that point. So,
% if LaTeX puts 3 of 4 authors names on the first line, and the last
% on the second line, try using \AND instead of \And before the third
% author name.

\author{
    Cody Ray Hoeft \\
    School of EECS \\
    Oregon State University \\
    Corvallis, OR 97331 \\
    \texttt{hoeftc@oregonstate.edu} \\
    \And
    Padraig Gillen \\
    School of EECS \\
    Oregon State University \\
    Corvallis, OR 97331 \\
    \texttt{gillenp@oregonstate.edu} \\
}

\begin{document}
% \nipsfinalcopy is no longer used

\maketitle

\begin{abstract}
    The website Quora helps anybody ask questions and answer those they have an understanding of.
    With a user base of...
\end{abstract}

\section{Introduction}
% An introduction section that briefly describes the problem you aimed to solve and a summary of the main results

\section{Approach}
% One / multiple sections describing the approach(es) you explored for solving the problem.
% Examples: feature design, preprocessing, and the prediction models etc.


\section{Results}
% section devoted to presenting / discussing the results  obtained from exploration
% possible discussion topics include:
%   what worked and what didn't?
%   How do different methods compare with each other?
%   what kind of lessons you learned from your exploration?
%   What are possible ways for future work to improve?

\section{Conclusion}
% A conclusion section that summarize your project and your effort.

\appendix
\section{Work Done}
% appendix describes individual contribution levels to the project
%    estimated in percentages and described in words

\end{document}

% \subsection{Citations within the text}
%
% The \verb+natbib+ package will be loaded for you by default.
% Citations may be author/year or numeric, as long as you maintain
% internal consistency.  As to the format of the references themselves,
% any style is acceptable as long as it is used consistently.
%
% The documentation for \verb+natbib+ may be found at
% \begin{center}
%     \url{http://mirrors.ctan.org/macros/latex/contrib/natbib/natnotes.pdf}
% \end{center}
% Of note is the command \verb+\citet+, which produces citations
% appropriate for use in inline text.  For example,
% \begin{verbatim}
%     \citet{hasselmo} investigated\dots
% \end{verbatim}
% produces
% \begin{quote}
%     Hasselmo, et al.\ (1995) investigated\dots
% \end{quote}
%
% If you wish to load the \verb+natbib+ package with options, you may
% add the following before loading the \verb+nips_2017+ package:
% \begin{verbatim}
%     \PassOptionsToPackage{options}{natbib}
% \end{verbatim}
%
% If \verb+natbib+ clashes with another package you load, you can add
% the optional argument \verb+nonatbib+ when loading the style file:
% \begin{verbatim}
%     \usepackage[nonatbib]{nips_2017}
% \end{verbatim}
%
% As submission is double blind, refer to your own published work in the
% third person. That is, use ``In the previous work of Jones et
% al.\ [4],'' not ``In our previous work [4].'' If you cite your other
% papers that are not widely available (e.g., a journal paper under
% review), use anonymous author names in the citation, e.g., an author
% of the form ``A.\ Anonymous.''
%
%
% \subsection{Tables}
%
% All tables must be centered, neat, clean and legible.  The table
% number and title always appear before the table.  See
% Table~\ref{sample-table}.
%
% Place one line space before the table title, one line space after the
% table title, and one line space after the table. The table title must
% be lower case (except for first word and proper nouns); tables are
% numbered consecutively.
%
% Note that publication-quality tables \emph{do not contain vertical
%   rules.} We strongly suggest the use of the \verb+booktabs+ package,
% which allows for typesetting high-quality, professional tables:
% \begin{center}
%   \url{https://www.ctan.org/pkg/booktabs}
% \end{center}
% This package was used to typeset Table~\ref{sample-table}.
%
% \begin{table}[t]
%   \caption{Sample table title}
%   \label{sample-table}
%   \centering
%   \begin{tabular}{lll}
%     \toprule
%     \multicolumn{2}{c}{Part}                   \\
%     \cmidrule{1-2}
%     Name     & Description     & Size ($\mu$m) \\
%     \midrule
%     Dendrite & Input terminal  & $\sim$100     \\
%     Axon     & Output terminal & $\sim$10      \\
%     Soma     & Cell body       & up to $10^6$  \\
%     \bottomrule
%   \end{tabular}
% \end{table}
